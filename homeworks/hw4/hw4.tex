\documentclass{article}
\usepackage{ctex}
\usepackage{amsmath}
\usepackage{amsthm}
\usepackage{verbatim}

\title{CODH作业4}
\author{张博厚 PB22071354}
\date{}
\setlength{\parindent}{0pt}

\begin{document}
\maketitle

\section*{4.16}
1.流水线: 350ps, 非流水线: 1250ps.\\
2.在两种处理器中, 对于ld指令的延迟均为1250ps.\\
3.拆分ID, 新处理器时钟周期为300ps.

\section*{4.23}
1.流水线级数的减少不一定会影响时钟周期, 这取决于延迟最长的流水线级是否改变.\\
2.可能提高性能. 将MEM和EX阶段重叠会减少一条指令执行完毕所需的时钟周期数, 同时可以
减少ld指令与使用其结果的R型指令之间的阻塞, 因此可能提高性能.\\
3.也可能降低性能. 使用寄存器作为访存地址而不需要立即数偏移, 可能会需要在原本的访存指令
前添加addi指令, 使得需要执行的总指令数增加.
\end{document}

