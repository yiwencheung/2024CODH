\documentclass{article}
\usepackage{ctex}
\usepackage{amsmath}
\usepackage{amsthm}
\usepackage{verbatim}

\title{CODH作业2}
\author{张博厚 PB22071354}
\date{}
\setlength{\parindent}{0pt}

\begin{document}
\maketitle
\section*{2.9}
1. opcode: 0010011,\ rs1: x10,\ rd: x30,\ imm: 8,\ funct3: 0x0\\
2. opcode: 0010011,\ rs1: x10,\ rd: x31,\ imm: 0,\ funct3: 0x0\\
3. opcode: 0100011,\ rs1: x31,\ rs2:x30,\ imm:0,\ funct3: 0x3\\
4. opcode: 0000011,\ rs1: x30,\ rs2:x30,\ imm:0,\ funct3: 0x3\\
5. opcode: 0110011,\ rs1: x30,\ rd: x5,\ rs2:x31,\ funct3:0x0,\ fucnt7:0x00

\section*{2.24}
1.最终值为20.\\
2.\begin{verbatim}
    int acc = 0, i=10;
    while(i){
        i--;
        acc += 2;
    }
\end{verbatim}
3.共执行了4n+1条指令.\\
4.\begin{verbatim}
    int acc = 0, i = 10;
    while(i >= 0){
        i--;
        acc += 2;
    }
\end{verbatim}

\section*{2.35}
1. 0x11\\
2. 0x88

\section*{2.40}
1.$CPI = 2\times0.7+ 6\times0.1 + 3\times0.2 = 2.6$\\
2.性能提高25\% 后, $CPI = 2.6\times0.75=1.95=x\times0.7+ 6\times0.1 + 3\times0.2 $,
解得$x=1.071$\\
3. $CPI = 2.6\times0.5=1.3=x\times0.7+ 6\times0.1 + 3\times0.2$,
解得$x=0.143$
\end{document}